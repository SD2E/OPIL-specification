% -----------------------------------------------------------------------------
\section{SBOL Imports: Identifiers, Primitive Types, and Classes}
% -----------------------------------------------------------------------------

OPIL builds on the Synthetic Biology Open Language (SBOL) version 3~\citep{SBOL3} in several ways:
\begin{itemize}
\item OPIL uses the same conventions as SBOL for URIs and types.
\item OPIL uses the SBOL base classes \sbol{Identified} and \sbol{TopLevel} as parents for OPIL classes.
\item OPIL uses SBOL classes to describe biological samples in terms of combinations of strains, media, etc.
\item OPIL makes use of the same external measurement ontology as SBOL, the Ontology of Units of Measure (OM) (\url{http://www.ontology-of-units-of-measure.org/resource/om-2}.
\end{itemize}

In order to make this document more self-contained, this section repeats the material from the SBOL specification on conventions and SBOL base classes.
A summary is also provided for key SBOL and OM classes; for complete documentation, see the SBOL specification.
In case of conflict between any material in this section and the SBOL specification, the SBOL specification should be held correct.

\subsection{Uniform Resource Identifiers}
\label{sec:URIstructure}

As OPIL is built upon the Resource Description Framework (RDF), all class instances are identified by a Uniform Resource Identifier (URI).  In the OPIL data model, the value of an association property MUST be a \opil{URI} or set of \opil{URI}s that refer to OPIL objects belonging to the class at the tip of the arrow.  Every \sbol{Identified} object's URI MUST be globally unique among all other \sbol{Identified} object URIs. It is also highly RECOMMENDED that the \opil{URI} structure follows the recommended best practices for compliant \opil{URI}s specified in \ref{sec:compliant}.

Whenever a \sbol{TopLevel} object's URI is a URL (e.g., following the conventions of HTTP(S) rather than a UUID), its structure MUST comply with the following rules:

\begin{itemize}

 \item A \sbol{TopLevel} URL MUST use the following pattern:
  \texttt{[namespace]/[local]/[displayId]},  where \texttt{namespace} and \sbol{displayId} are required fragments, and the \texttt{local} fragment is an optional relative path.
  
  	For example, a \sbol{Component} might have the URL~\path{https://synbiohub.org/public/igem/BBa_J23070}, where \texttt{namespace} is \path{https://synbiohub.org}, \texttt{local} is \path{public/igem}, and \texttt{displayId} is \path{BBa_J23070}.

  \item A \sbol{TopLevel} object's URL MUST NOT be included as prefix for any other \sbol{TopLevel} object.
  
  	For example, the \path{BBa_J23070_seq} \sbol{Sequence} object cannot have a URL of \path{https://synbiohub.org/public/igem/BBa_J23070/BBa_J23070_seq}, since the \path{https://synbiohub.org/public/igem/BBa_J23070} prefix is already used as a URL for the \path{BBa_J23070} \sbol{Component} object.

  \item The URL of any child or nested object MUST use the following pattern:\texttt{[parent]/[displayId]}, where \texttt{parent} is the URL of its parent object.
	Multiple layers of child objects are allowed using the same\\ \texttt{[parent]/[displayId]} pattern recursively.
	
	For example, a \sbol{SequenceFeature} object owned by the \path{BBa_J23070} \sbol{Component} and having a \sbol{displayId} of \texttt{SequenceFeature1} will have a URL of \path{https://synbiohub.org/public/igem/BBa_J23070/SequenceFeature1}.
	Similarly, if the \texttt{SequenceFeature1} object has a \sbol{Location} child object with a \sbol{displayId} of \texttt{Location1}, then that object will have the URL\\ \path{https://synbiohub.org/public/igem/BBa_J23070/SequenceFeature1/Location1}.
  \end{itemize}

\subsubsection{Compliant URIs}
\label{sec:compliant}

Maintaining unique URIs for objects can be challenging.  Compliant URIs follow a set of rules that simplify this challenge.

Compliant URIs for \sbol{TopLevel} objects MUST conform to the following pattern:
\begin{quotation} 
\refObj{namespace}/\refObj{collection\_structure}/\refObj{displayId}
\end{quotation}

The \refObj{namespace} token MAY further decompose into \refObj{domain}/\refObj{root} tokens. The \refObj{root} and \refObj{collection\_structure} tokens may optionally be omitted; alternatively, they may consist of an arbitrary number of delimiter-separated layers. Note that this pattern means that SBOL-compliant \opil{URI}s can be automatically decomposed with the aid of a \sbol{TopLevel} object's \sbol{hasNamespace} property. SBOL-compliant objects can be easily remapped into new namespaces by changing only the \refObj{namespace}.

Consider, for example, the SBOL-compliant \opil{URI}:
\begin{quote}``https://synbiohub.org/igem/2017\_distribution/promoters/constitutive/BBa\_J23101''\end{quote} 
for a \sbol{Component} with a \sbol{hasNamespace} value ``https://synbiohub.org/igem/2017\_distribution''.
This \opil{URI} can be decomposed as follows:
\begin{quote} 
namespace: ``https://synbiohub.org/igem/2017\_distribution'' \linebreak
domain: ``https://synbiohub.org'' \linebreak
root: ``igem/2017\_distribution'' \linebreak
collection: ``promoters/constitutive'' \linebreak
displayId: ``BBa\_J23101'' \linebreak
\end{quote}

SBOL-compliant URIs also facilitate auto-construction of child objects with unique \opil{URI}s. 
Child objects of \sbol{TopLevel} objects with compliant \opil{URI}s MUST conform to the following pattern:\\ ``\refObj{parent\_uri}/\refObj{child\_type}\refObj{child\_type\_counter}'' where the \refObj{parent\_uri} refers to the URI of the parent object, the \refObj{child\_type} refers to the SBOL class of the child object, and \refObj{child\_type\_counter} is a unique index for the child object. 
The \refObj{child\_type\_counter} of a new object SHOULD be calculated at time of object creation as 1 + the maximum \refObj{child\_type\_counter} for each \refObj{child\_type} object in the parent (e.g., ``\refObj{parent\_uri}/SequenceAnnotation37''). 
Note that numbering is independent for each type, so a \sbol{Component} can have children ``SubComponent37'' and ``Constraint37''.

\subsection{OPIL URIs}
 \label{sec:sbolURIs}
  
\todo[inline]{This needs to be changed from BBN to a more general community name and given a version number}
The SBOL namespace, which is \url{http://bbn.com/synbio/opil\#}, is used to indicate which entities and properties in an OPIL document are defined by OPIL. 
For example, the URI of the type \opil{ProtocolInderface} is \url{http://bbn.com/synbio/opil\#ProtocolInterface}. 
This convention is assumed throughout the specification.
The OPIL namespace MUST NOT be used for any entities or properties not defined in this specification.  

Other namespaces are also used by OPIL, however, notably the SBOL 3 namespace \url{http://sbols.org/v3\#}, as well as other namespaces used by SBOL including Dublin Core~\citep{dcmi2012}), Ontology of Units of Measure (OM), and various biological ontologies.


\subsection{Primitive Data Types}
\label{sec:datatypes}
\label{sec:String}
\label{sec:Integer}
\label{sec:Long}
\label{sec:Double}
\label{sec:Boolean}
\label{sec:URI}
\label{sec:literal}

When OPIL uses simple ``primitive'' data types such as \opil{String}s or \opil{Integer}s, these are defined as the following specific formal types:
\begin{itemize}
\item \opil{String}: \url{http://www.w3.org/2001/XMLSchema\#string}\\
  {\em Example: ``LacI coding sequence''}
\item \opil{Integer}: \url{http://www.w3.org/2001/XMLSchema\#integer}\\
  {\em Example: 3}
\item \opil{Long}: \url{http://www.w3.org/2001/XMLSchema\#long}\\
  {\em Example: 9223372036854775806}
\item \opil{Double}: \url{http://www.w3.org/2001/XMLSchema\#double}\\
  {\em Example: 3.14159}
\item \opil{Boolean}: \url{http://www.w3.org/2001/XMLSchema\#boolean}\\
  {\em Example: \external{true}}
\end{itemize}

The term \opil{literal} is used to denote an object that can be any of the five types listed above.

In addition to the simple types listed above, SBOL also uses objects with types \emph{Uniform Resource Identifier} (\opil{URI}). It is important to realize that in RDF, a \opil{URI} might or might not be a resolvable URL (web address).  A \opil{URI} is always a globally unique identifier within a structured namespace.  In some cases, that name is also a reference to (or within) a document, and in some cases that document can also be retrieved (e.g., using a web browser).

\subsection{OPIL Types}
\label{sec:sbolTypes}

\todo[inline]{Change to match above when root namespace changes: perhaps bioprotocols.org?}
All OPIL objects are given the most specific \external{rdfType} in the OPIL namespace (\uri{http://bbn.com/synbio/opil\#}) that defines the type of the object.  Namely, an object MUST have no more than one \external{rdfType} property in the \uri{http://bbn.com/synbio/opil\#} namespace.

For example, an object cannot have both the \external{rdfType} of \opil{ProtocolInterface} and \opil{MeasurementType}.  Also, a \opil{MeasureParameter} would have this \external{rdfType} and not also include \external{rdfType}s for classes that it inherits from, such as \sbol{Parameter} and \sbol{Identified}.

