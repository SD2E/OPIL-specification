\section{Serialization}
\label{sec:serialization}

In order for OPIL objects to be readily stored and exchanged, it is important that they are able to be {\em serialized}, i.e., converted to a sequence of bytes that can be stored in a file or exchanged over a network.  
%
To this end, OPIL builds upon the Resource Description Framework (RDF).  RDF is an abstract language for describing conceptual graph-oriented data models, and therefore does not mandate any specific serialization format.  Instead, a number of different serialization formats are provided as separate specifications, such as RDF/XML, N-Triples, JSON-LD, and Turtle.  These serialization formats are widely supported by RDF libraries such as rdflib for Python and Apache Jena for Java.

All OPIL libraries SHOULD support at least RDF/XML, N-Triples, JSON-LD, and Turtle.
Other OPIL tools SHOULD support at least one of these four formats.
